\documentclass[10pt,a4paper,ragged2e,withhyper]{altacv}

%% AltaCV uses the fontawesome5 package.
%% See http://texdoc.net/pkg/fontawesome5 for full list of symbols.

% Change the page layout if you need to
\geometry{left=1.25cm,right=1.25cm,top=1.5cm,bottom=1.5cm,columnsep=1.2cm}

% The paracol package lets you typeset columns of text in parallel
\usepackage{paracol}
\usepackage{url}


% Change the font if you want to, depending on whether
% you're using pdflatex or xelatex/lualatex
\ifxetexorluatex
  % If using xelatex or lualatex:
  \setmainfont{Lato}
\else
  % If using pdflatex:
  \usepackage[default]{lato}
\fi

% Change the colours if you want to
\definecolor{VividPurple}{HTML}{3E0097}
\definecolor{SlateGrey}{HTML}{2E2E2E}
\definecolor{LightGrey}{HTML}{666666}
% \colorlet{name}{black}
% \colorlet{tagline}{PastelRed}
\colorlet{heading}{VividPurple}
\colorlet{headingrule}{VividPurple}
% \colorlet{subheading}{PastelRed}
\colorlet{accent}{VividPurple}
\colorlet{emphasis}{SlateGrey}
\colorlet{body}{LightGrey}

% Change some fonts, if necessary
% \renewcommand{\namefont}{\Huge\rmfamily\bfseries}
% \renewcommand{\personalinfofont}{\footnotesize}
% \renewcommand{\cvsectionfont}{\LARGE\rmfamily\bfseries}
% \renewcommand{\cvsubsectionfont}{\large\bfseries}

% Change the bullets for itemize and rating marker
% for \cvskill if you want to
\renewcommand{\itemmarker}{{\small\textbullet}}
\renewcommand{\ratingmarker}{\faCircle}

%% Use (and optionally edit if necessary) this .tex if you
%% want to use an author-year reference style like APA(6)
%% for your publication list
\input{pubs-authoryear}

%% Use (and optionally edit if necessary) this .tex if you
%% want an originally numerical reference style like IEEE
%% for your publication list
% \input{pubs-num}

%% sample.bib contains your publications
\addbibresource{sample.bib}

\begin{document}
\name{Jonathan Skopp}
\tagline{Student der Informatik}
% Cropped to square from https://en.wikipedia.org/wiki/Marissa_Mayer#/media/File:Marissa_Mayer_May_2014_(cropped).jpg, CC-BY 2.0
%% You can add multiple photos on the left or right
%%\photoR{2.5cm}{mmayer-wikipedia-cc-by-2_0}
% \photoL{2cm}{Yacht_High,Suitcase_High}
\personalinfo{%
  % Not all of these are required!
  % You can add your own with \printinfo{symbol}{detail}
  \email{jonathan.skopp@gmail.com}
%   \phone{000-00-0000}
  \mailaddress{Weimarer Straße 81, 98693 Ilmenau}
  \linkedin{jonathan-skopp-556a231a1}\\
  \github{Riffecs}
  \twitter{jskopp98}

}

\makecvheader

%% Depending on your tastes, you may want to make fonts of itemize environments slightly smaller
\AtBeginEnvironment{itemize}{\small}

%% Set the left/right column width ratio to 6:4.
\columnratio{0.6}

% Start a 2-column paracol. Both the left and right columns will automatically
% break across pages if things get too long.
\begin{paracol}{2}

\cvsection{Erfahrungen}

\cvevent{Besuch ausgewählter Module der Mathematik
}{Fernuniversität Hagen
}{April 2020 – April 2022 }{Hagen, NRW}
\begin{itemize}
\item Einschreibung in den Bachelor Studiengang Mathematik
\item Besuch diverser mathematischer Praktika
\end{itemize}


\divider

\cvevent{Abitur mit Schwerpunkt Datenverarbeitung
}{Andreas Gordon Kompetenzzentrum Erfurt
}{Juli 2015 – Juli 2018}{Erfurt, THÜ}
\begin{itemize}
\item Wahl des Nebenfaches Datenverarbeitung
\item Erstellung eines Programms zur asymmetrischen Verschlüsselung von
Dateien (Java)
\item Vermittlung grundlegender Elemente des 3D-Drucks
\item Vermittlung grundlegender Elemente der Softwarearchitektur
\end{itemize}

\divider

\cvevent{Ehrenamtlicher Vorstand der Thüringer Schachjugend
}{Thüringer Schachjugend}{Februar 2021 – März 2022}{Erfurt, THÜ}
\begin{itemize}
\item Ausrichten der Thüringer Meisterschaft
\item Verwalten des Kinder- und Jugendtrainings im Land

\end{itemize}

\divider

\cvevent{Ehrenamtliches Kinder- und Jugendtraining}{SV Empor Erfurt e.V., Abt. Schach
}{April 2015 – April 2019 }{Erfurt, THÜ}
\begin{itemize}
\item Betreuung von Mannschaften bei der deutschen Schulmeisterschaft
\item Grundlagenausbildung im Leistungssport Schach
\item Erreichen einer C-Trainer Ausbildung im Schach sowie einer\\ \mbox{Schiedsrichterlizenz}
\end{itemize}

\divider



\cvevent{Verwaltung des Streams von GM Niclas Huschenbeth
}{Twitch / Discord
}{Juli 2020 – April 2022}{online}
\begin{itemize}
\item Moderation des Streamchats
\item Verwaltung und Erstellung von SubBattles
\item Verwaltung von Servern und einer Cloud
\end{itemize}


%% Switch to the right column. This will now automatically move to the second
%% page if the content is too long.
\switchcolumn

\cvsection{Lebensweisheit}
\begin{quote}
``We don’t have to know what tomorrow holds! That’s why we can live for everything we’re worth today.''
\end{quote}


\cvsection{Skills}

\cvtag{Java}
\cvtag{Rust}
\cvtag{Python}
\cvtag{Julia}
\cvtag{Go}
\cvtag{C++}
\cvtag{C\#}
\cvtag{Matlab}
\cvtag{NodeJS}
\cvtag{\LaTeX}


\divider\smallskip

\cvtag{Führerschein}
\cvtag{Microsoft Office}\\
\cvtag{C-Trainer Schach}


\cvsection{Sprachen}

\cvskill{Deutsch}{5}
% \divider

\cvskill{Englisch}{2} %% supports X.5 values.


\cvsection{Werdegang}

\cvevent{Studium Informatik}{Technische Universität Ilmenau}{Oktober 2018 – *}{}

\divider

\cvevent{Abitur mit Schwerpunkt Datenverarbeitung}{Andreas Gordon Kompetenzzentrum Erfurt
}{Juli 2015 – Juli 2018}{}

\cvsection{Veröffentlichungen}
\begin{itemize}
    \item Grundlagen der Kryptografie (Seminarfacharbeit)
    %% \item Summe der natürlichen Zahlen 
\end{itemize}

\cvsection{Softwareprojekte}

\printbibliography[heading=pubtype,title={\printinfo{\faFile*[regular]}{Softwareprojekte}}, type=article]


\begin{itemize}
    \item Python Lichess Wrapper
    \begin{itemize}
        \item \url{https://github.com/Riffecs/lichesspy}
    \end{itemize}
    \item Java Crypt Module (RSA)
    \begin{itemize}
        \item \url{https://github.com/Riffecs/JavaCrypt}
    \end{itemize}
    \item Python Package Manager für die X-FAB (Client und Server)
    \begin{itemize}
        \item \url{https://github.com/Riffecs/rion}
         \item \url{https://github.com/Riffecs/inor}
    \end{itemize}
    \item Simple Unity (C\#) Game (Work in progress)
    \begin{itemize}
        \item \url{https://github.com/Riffecs/Griff}
    \end{itemize}
    
\end{itemize}

\end{paracol}

\end{document}
